% generated by GAPDoc2LaTeX from XML source (Frank Luebeck)
\documentclass[a4paper,11pt]{report}

\usepackage[top=37mm,bottom=37mm,left=27mm,right=27mm]{geometry}
\sloppy
\pagestyle{myheadings}
\usepackage{amssymb}
\usepackage[utf8]{inputenc}
\usepackage{makeidx}
\makeindex
\usepackage{color}
\definecolor{FireBrick}{rgb}{0.5812,0.0074,0.0083}
\definecolor{RoyalBlue}{rgb}{0.0236,0.0894,0.6179}
\definecolor{RoyalGreen}{rgb}{0.0236,0.6179,0.0894}
\definecolor{RoyalRed}{rgb}{0.6179,0.0236,0.0894}
\definecolor{LightBlue}{rgb}{0.8544,0.9511,1.0000}
\definecolor{Black}{rgb}{0.0,0.0,0.0}

\definecolor{linkColor}{rgb}{0.0,0.0,0.554}
\definecolor{citeColor}{rgb}{0.0,0.0,0.554}
\definecolor{fileColor}{rgb}{0.0,0.0,0.554}
\definecolor{urlColor}{rgb}{0.0,0.0,0.554}
\definecolor{promptColor}{rgb}{0.0,0.0,0.589}
\definecolor{brkpromptColor}{rgb}{0.589,0.0,0.0}
\definecolor{gapinputColor}{rgb}{0.589,0.0,0.0}
\definecolor{gapoutputColor}{rgb}{0.0,0.0,0.0}

%%  for a long time these were red and blue by default,
%%  now black, but keep variables to overwrite
\definecolor{FuncColor}{rgb}{0.0,0.0,0.0}
%% strange name because of pdflatex bug:
\definecolor{Chapter }{rgb}{0.0,0.0,0.0}
\definecolor{DarkOlive}{rgb}{0.1047,0.2412,0.0064}


\usepackage{fancyvrb}

\usepackage{mathptmx,helvet}
\usepackage[T1]{fontenc}
\usepackage{textcomp}


\usepackage[
            pdftex=true,
            bookmarks=true,        
            a4paper=true,
            pdftitle={Written with GAPDoc},
            pdfcreator={LaTeX with hyperref package / GAPDoc},
            colorlinks=true,
            backref=page,
            breaklinks=true,
            linkcolor=linkColor,
            citecolor=citeColor,
            filecolor=fileColor,
            urlcolor=urlColor,
            pdfpagemode={UseNone}, 
           ]{hyperref}

\newcommand{\maintitlesize}{\fontsize{50}{55}\selectfont}

% write page numbers to a .pnr log file for online help
\newwrite\pagenrlog
\immediate\openout\pagenrlog =\jobname.pnr
\immediate\write\pagenrlog{PAGENRS := [}
\newcommand{\logpage}[1]{\protect\write\pagenrlog{#1, \thepage,}}
%% were never documented, give conflicts with some additional packages

\newcommand{\GAP}{\textsf{GAP}}

%% nicer description environments, allows long labels
\usepackage{enumitem}
\setdescription{style=nextline}

%% depth of toc
\setcounter{tocdepth}{1}





%% command for ColorPrompt style examples
\newcommand{\gapprompt}[1]{\color{promptColor}{\bfseries #1}}
\newcommand{\gapbrkprompt}[1]{\color{brkpromptColor}{\bfseries #1}}
\newcommand{\gapinput}[1]{\color{gapinputColor}{#1}}


\begin{document}

\logpage{[ 0, 0, 0 ]}
\begin{titlepage}
\mbox{}\vfill

\begin{center}{\maintitlesize \textbf{ GaussPar \mbox{}}}\\
\vfill

\hypersetup{pdftitle= GaussPar }
\markright{\scriptsize \mbox{}\hfill  GaussPar  \hfill\mbox{}}
{\Huge \textbf{ Parallel gaussian algorithm for finite fields \mbox{}}}\\
\vfill

{\Huge  1.0.0 \mbox{}}\\[1cm]
{ 5 May 2021 \mbox{}}\\[1cm]
\mbox{}\\[2cm]
{\Large \textbf{ Jendrik Brachter\\
  \mbox{}}}\\
{\Large \textbf{ Sergio Siccha\\
  \mbox{}}}\\
{\Large \textbf{ Emma Ahrens\\
  \mbox{}}}\\
\hypersetup{pdfauthor= Jendrik Brachter\\
  ;  Sergio Siccha\\
  ;  Emma Ahrens\\
  }
\end{center}\vfill

\mbox{}\\
{\mbox{}\\
\small \noindent \textbf{ Jendrik Brachter\\
  }  Email: \href{mailto://brachter@cs.uni-kl.de} {\texttt{brachter@cs.uni-kl.de}}}\\
{\mbox{}\\
\small \noindent \textbf{ Sergio Siccha\\
  }  Email: \href{mailto://siccha@mathematik.uni-kl.de} {\texttt{siccha@mathematik.uni-kl.de}}}\\
{\mbox{}\\
\small \noindent \textbf{ Emma Ahrens\\
  }  Email: \href{mailto://emma.ahrens@rwth-aachen.de} {\texttt{emma.ahrens@rwth-aachen.de}}}\\
\end{titlepage}

\newpage\setcounter{page}{2}
\newpage

\def\contentsname{Contents\logpage{[ 0, 0, 1 ]}}

\tableofcontents
\newpage

     
\chapter{\textcolor{Chapter }{The GaussPar package}}\label{Chapter_The_GaussPar_package}
\logpage{[ 1, 0, 0 ]}
\hyperdef{L}{X84544BE787523F80}{}
{
  
\section{\textcolor{Chapter }{Introduction}}\label{Chapter_The_GaussPar_package_Section_Introduction}
\logpage{[ 1, 1, 0 ]}
\hyperdef{L}{X7DFB63A97E67C0A1}{}
{
  This package implements the algorithm described in "A parallel algorithm for
Gaussian elimination over finite fields" (Linton, S., Nebe, G., Niemeyer, A.,
Parker, R. and Thackray, J. (2018)). The algorithm divides a given matrix into
smaller submatrices (blocks) and carries out steps of the Gaussian elimination
block by block. The paper identifies the minimal dependencies between
processing different blocks. Independent steps can be performed
simultaneously, allowing for high levels of parallelism. We provide two main
functions \texttt{EchelonMatBlockwise} (\ref{EchelonMatBlockwise}) and \texttt{EchelonMatTransformationBlockwise} (\ref{EchelonMatTransformationBlockwise}), where the latter additionally computes the transformation matrix in
parallel. The naming conventions are chosen in accordance with the
"Gauss"-package 
\footnote{\url{https://www.gap-system.org/Packages/gauss.html}}
 , which provides an implementation of the sequential GAUSS algorithm. Our
functions do currently not support sparse matrices. 

 }

 }

   
\chapter{\textcolor{Chapter }{Gaussian Elimination}}\label{Chapter_Gaussian_Elimination}
\logpage{[ 2, 0, 0 ]}
\hyperdef{L}{X84FE9E817C8A74B1}{}
{
  
\section{\textcolor{Chapter }{Gaussian Elimination}}\label{Chapter_Gaussian_Elimination_Section_Gaussian_Elimination}
\logpage{[ 2, 1, 0 ]}
\hyperdef{L}{X84FE9E817C8A74B1}{}
{
  

 This section describes the different variants of our implementation of the
Gaussian algorithm. 

 Note that the value of the option \mbox{\texttt{\mdseries\slshape numberBlocks}} described below has a big impact on the performance of our algorithms. For
information on how to choose a suitable value for \mbox{\texttt{\mdseries\slshape numberBlocks}} see Chapter \ref{Chapter_Finding_a_suitable_number_of_blocks}. 

 

\subsection{\textcolor{Chapter }{EchelonMatTransformationBlockwise}}
\logpage{[ 2, 1, 1 ]}\nobreak
\hyperdef{L}{X86CC1A2A7832BB22}{}
{\noindent\textcolor{FuncColor}{$\triangleright$\enspace\texttt{EchelonMatTransformationBlockwise({\mdseries\slshape mat[, options]})\index{EchelonMatTransformationBlockwise@\texttt{EchelonMatTransformationBlockwise}}
\label{EchelonMatTransformationBlockwise}
}\hfill{\scriptsize (function)}}\\
\textbf{\indent Returns:\ }
a record that contains information on the echelon form of \emph{mat} and the corresponding transformation matrix. 



 This is the main function of the GaussPar package. It computes the reduced row
echelon form (RREF) of the matrix \emph{mat} and the corresponding transformation matrix. In a pre-processing step, \emph{mat} is split up into a block matrix whose blocks can be processed in parallel. 

 The input parameters have the following meaning: 
\begin{itemize}
\item  \emph{mat} is a matrix defined over a finite field 
\item  \emph{options} is a record that can be used to provide some additional parameters. Note that
a specification of either \emph{numberBlocks} or both of \emph{numberBlocksHeight} and \emph{numberBlocksWidth} is mandatory. The following parameters are currently supported: 
\begin{itemize}
\item  \emph{numberBlocksHeight} and \emph{numberBlocksWidth}: The number of vertical and horizontal blocks in which to divide the matrix
during the algorithm. 
\item  \emph{numberBlocks}: Use this argument if you want the same number of vertical and horizontal
blocks in the block matrix decomposition of \emph{mat}. 
\item  \emph{verify}: If set to \emph{true}, the computation is verified at the end. That is, we check wheter \emph{coeffs} * \emph{mat} is in RREF. This option is only available for the function \emph{EchelonMatTransformationBlockwise}. 
\item  \emph{isChopped}: It is possible to input \emph{mat} directly as a matrix of block matrices. In this case the parameter \emph{isChopped} must be set to \emph{true} and the splitting step is skipped. 
\end{itemize}
 
\end{itemize}
 

 The output record contains the following items: 
\begin{itemize}
\item  \emph{vectors}: a matrix that forms the RREF of \emph{mat} without zero rows 
\item  \emph{heads}: a list of integers, such that \emph{heads[i]} gives the number of the row for which the pivot element is in column i. If no
such row exists, \emph{heads[i]} is \emph{0}. 
\item  \emph{coeffs}: the corresponding transformation matrix. It holds \emph{coeffs} * \emph{mat} = \emph{vectors}. 
\item  \emph{relations}: a matrix whose rows form a basis for the row null space of \emph{mat}. If \emph{relations} is not the empty list, it holds that \emph{relations} * \emph{mat} = \emph{0}. Otherwise \emph{mat} has full row rank. 
\end{itemize}
 

 
\begin{Verbatim}[commandchars=!@|,fontsize=\small,frame=single,label=Example]
  !gapprompt@gap>| !gapinput@A := RandomMat(8, 5, GF(5)) * RandomMat(5, 8, GF(5));;|
  !gapprompt@gap>| !gapinput@Display(A);|
  1 4 3 2 4 4 3 4
  4 1 2 4 2 . . 4
  2 3 1 4 3 3 1 3
  3 . 4 3 3 2 4 .
  4 1 3 2 3 3 . 2
  2 1 3 3 1 1 2 3
  . 3 3 . 1 1 3 .
  4 1 4 1 4 3 1 1
  !gapprompt@gap>| !gapinput@res := EchelonMatTransformationBlockwise(A, rec(numberBlocks := 2));;|
  !gapprompt@gap>| !gapinput@Display(res.vectors);|
  1 . . . . . . 3
  . 1 . . . 3 2 3
  . . 1 . . . . 3
  . . . 1 . 2 1 1
  . . . . 1 2 2 2
  !gapprompt@gap>| !gapinput@res.coeffs * A=res.vectors;|
  true
\end{Verbatim}
 The transformation matrix can be easily obtained from the output record as
follows: 
\begin{Verbatim}[commandchars=!@|,fontsize=\small,frame=single,label=Example]
  !gapprompt@gap>| !gapinput@trafo := Concatenation(res.coeffs, res.relations);;|
  !gapprompt@gap>| !gapinput@Display(trafo * A);|
  1 . . . . . . 3
  . 1 . . . 3 2 3
  . . 1 . . . . 3
  . . . 1 . 2 1 1
  . . . . 1 2 2 2
  . . . . . . . .
  . . . . . . . .
  . . . . . . . .
\end{Verbatim}
 It is also possible to directly input a block matrix. 
\begin{Verbatim}[commandchars=!@|,fontsize=\small,frame=single,label=Example]
  !gapprompt@gap>| !gapinput@A := RandomMat(8, 5, GF(5)) * RandomMat(5, 8, GF(5));;|
  !gapprompt@gap>| !gapinput@res1 := EchelonMatTransformationBlockwise(A, rec(numberBlocks := 2));;|
  !gapprompt@gap>| !gapinput@A1 := A{[1..4]}{[1..4]};;|
  !gapprompt@gap>| !gapinput@A2 := A{[1..4]}{[5..8]};;|
  !gapprompt@gap>| !gapinput@A3 := A{[5..8]}{[1..2]};;|
  !gapprompt@gap>| !gapinput@A4 := A{[5..8]}{[5..8]};;|
  !gapprompt@gap>| !gapinput@A_blockwise := [[A1,A2],[A3,A4]];;|
  !gapprompt@gap>| !gapinput@res3 := EchelonMatTransformationBlockwise(A_blockwise, |
    rec(numberBlocks := 2, isChopped := true));;
  !gapprompt@gap>| !gapinput@res3 = res1;|
  true
\end{Verbatim}
 }

 

\subsection{\textcolor{Chapter }{EchelonMatBlockwise}}
\logpage{[ 2, 1, 2 ]}\nobreak
\hyperdef{L}{X7DFFD3D985EF8101}{}
{\noindent\textcolor{FuncColor}{$\triangleright$\enspace\texttt{EchelonMatBlockwise({\mdseries\slshape mat[, options]})\index{EchelonMatBlockwise@\texttt{EchelonMatBlockwise}}
\label{EchelonMatBlockwise}
}\hfill{\scriptsize (function)}}\\
\textbf{\indent Returns:\ }
a record that contains information on the echelon form of \emph{mat}. 



 This is a version of the main function that computes the reduced row echelon
form (RREF) of the matrix \emph{mat} but doesn't compute the corresponding transformation matrix. In a
pre-processing step, \emph{mat} is split up into a block matrix whose blocks can be processed in parallel. 

 The input parameters have the following meaning: 
\begin{itemize}
\item  \emph{mat} is a matrix defined over a finite field 
\item  \emph{options} is a record that is used to provide additional parameters. For more
information see \emph{EchelonMatTransformationBlockwise} (See \texttt{EchelonMatTransformationBlockwise} (\ref{EchelonMatTransformationBlockwise})). 
\end{itemize}
 

 The output record contains the items \emph{vectors} and \emph{heads}. For their meaning, see \texttt{EchelonMatTransformationBlockwise} (\ref{EchelonMatTransformationBlockwise}). }

 }

 }

   
\chapter{\textcolor{Chapter }{Finding a suitable number of blocks}}\label{Chapter_Finding_a_suitable_number_of_blocks}
\logpage{[ 3, 0, 0 ]}
\hyperdef{L}{X82FE947A7FD1801E}{}
{
  

 Experiments with matrices over fields of sizes 2 to 11 and dimensions 500 to
10.000 have found values for \mbox{\texttt{\mdseries\slshape numberBlocks}} from 6 to 15 to be acceptable. Note though, that this highly depends on the
calculation, the number of used threads and the machine itself. 

 
\section{\textcolor{Chapter }{Measure Contention}}\label{Chapter_Finding_a_suitable_number_of_blocks_Section_Measure_Contention}
\logpage{[ 3, 1, 0 ]}
\hyperdef{L}{X79ABEDA87A10C6C2}{}
{
  

 

\subsection{\textcolor{Chapter }{GAUSS{\textunderscore}MeasureContention}}
\logpage{[ 3, 1, 1 ]}\nobreak
\hyperdef{L}{X85BB04B087C03C20}{}
{\noindent\textcolor{FuncColor}{$\triangleright$\enspace\texttt{GAUSS{\textunderscore}MeasureContention({\mdseries\slshape numberBlocks, q, A[, showOutput]})\index{GAUSSuScoreMeasureContention@\texttt{GAU}\-\texttt{S}\-\texttt{S{\textunderscore}}\-\texttt{Measure}\-\texttt{Contention}}
\label{GAUSSuScoreMeasureContention}
}\hfill{\scriptsize (function)}}\\


 

 This function helps with finding a suitable value for \mbox{\texttt{\mdseries\slshape numberBlocks}}, which has a big influence on the performance of the \texttt{EchelonMatTransformationBlockwise} (\ref{EchelonMatTransformationBlockwise}) function. For inputs \mbox{\texttt{\mdseries\slshape numberBlocks}}, a field size \mbox{\texttt{\mdseries\slshape q}}, and a matrix \mbox{\texttt{\mdseries\slshape A}} over \texttt{GF(q)}, this function does the following: 
\begin{itemize}
\item  it calls the parallel function \texttt{EchelonMatTransformationBlockwise} (\ref{EchelonMatTransformationBlockwise}), 
\item  it calls the sequential function \texttt{EchelonMatTransformation} from the package \texttt{Gauss}, 
\item  it prints the wall time, that is the elapsed real time, each call took, 
\item  for the parallel function, it prints an estimate of the lock contenation
ratio, a short explanation of lock contention is given below. A high lock
contention ratio deteriorates the performance of the algorithm. 
\end{itemize}
 

 The input \mbox{\texttt{\mdseries\slshape showOutput}} can be used to suppress printing of messages. 

 The influence of \mbox{\texttt{\mdseries\slshape numberBlocks}} on the performance is as follows: 
\begin{itemize}
\item  if \mbox{\texttt{\mdseries\slshape numberBlocks}} is too small, then not enough calculations can happen in parallel 
\item  if \mbox{\texttt{\mdseries\slshape numberBlocks}} is too big then: 
\begin{itemize}
\item  the lock contention ratio increases 
\item  HPC-GAP's task framework generates a big overhead since it is not optimized. 
\end{itemize}
 
\end{itemize}
 

 In a parallel computation several threads may try to access an object at the
same time. HPC-GAP needs to prevent situations in which one thread writes into
an object that another thread is currently reading, since that could lead to
corrupted data. To this end, HPC-GAP can synchronize access to objects via
locks. 

 If a thread reads an object, that thread can acquire a lock for that object,
that is no other thread can write into it until the lock is released. If a
thread tries to write into said object before the lock is released we say that
the lock was contended. In such a situation the contending thread needs to
wait until the lock is released. This leads to waiting times or unnecessary
context-switches and deteriorates performance if it happens too often. 

 
\begin{Verbatim}[commandchars=@|D,fontsize=\small,frame=single,label=Example]
  @gapprompt|gap>D @gapinput|n := 4000;; numberBlocks := 8;; q := 5;;D
  @gapprompt|gap>D @gapinput|A := RandomMat(n, n, GF(q));;D
  @gapprompt|gap>D @gapinput|GAUSS_MeasureContention(numberBlocks, q, A);D
  Make sure you called GAP with sufficient preallocated memory via `-m` if you
  try non-trivial examples! Otherwise garbage collection will be a big
  overhead.
  
  Starting the parallel algorithm
  EchelonMatTransformationBlockwise.
  
  Wall time  parallel execution (ms): 33940.
  CPU  time  parallel execution (ms): 2
  Lock statistics(estimates):
  acquired - 181502, contended - 1120, ratio - 1.%
  Locks acquired and contention counters per thread
  [ thread, locks acquired, locks contended ]:
  [ 5, 54093, 248 ]
  [ 6, 51004, 228 ]
  [ 7, 37102, 389 ]
  [ 8, 39303, 255 ]
  
  Starting the sequential algorithm
  EchelonMatTransformation
  
  Wall time Gauss pkg execution (ms): 66778.
  
  Speedup factor (sequential / parallel wall time):
  1.96
\end{Verbatim}
 }

 }

 }

   
\chapter{\textcolor{Chapter }{Using the task framework provided by HPC-GAP}}\label{Chapter_Using_the_task_framework_provided_by_HPC-GAP}
\logpage{[ 4, 0, 0 ]}
\hyperdef{L}{X7CF339FF7F595580}{}
{
  

 To implement our parallel version of the Gauss algorithm we use the task
framework provided by HPC-GAP. The structure of the source files reflects this
by grouping our functions depending on how they make use of HPC-GAP's shared
memory model. 

 
\section{\textcolor{Chapter }{The Package's Structure}}\label{Chapter_Using_the_task_framework_provided_by_HPC-GAP_Section_The_Packages_Structure}
\logpage{[ 4, 1, 0 ]}
\hyperdef{L}{X7D9749A5807B1B4C}{}
{
  

 
\begin{itemize}
\item  \texttt{main.gi}: Contains the main function \texttt{DoEchelonMatTransformationBlockwise}, which is wrapped by \texttt{EchelonMatBlockwise} (\ref{EchelonMatBlockwise}) and \texttt{EchelonMatTransformationBlockwise} (\ref{EchelonMatTransformationBlockwise}). It is the function which schedules all tasks. 
\end{itemize}
 

 
\begin{itemize}
\item  \texttt{dependencies.g}: The functions in this file compute the dependencies of the algorithm's
subprograms between each other. 
\end{itemize}
 

 
\begin{itemize}
\item  \texttt{tasks.g}: The functions in this file are scheduled as tasks by the main routine. They
need to make sure that they only write read-only objects into the "shared"
atomic lists. 
\end{itemize}
 

 
\begin{itemize}
\item  \texttt{thread-local.g}: The functions in this file are called by functions from "tasks.g". In
principle, these functions only work in a single thread-local region and don't
need to know anything about other threads. These functions may only access
read-only objects or objects from the executing thread's thread-local region.
They may only emit or write into thread-local objects. 
\end{itemize}
 

 }

 }

   
\chapter{\textcolor{Chapter }{Utility Functions}}\label{Chapter_Utility_Functions}
\logpage{[ 5, 0, 0 ]}
\hyperdef{L}{X810FFB1C8035C8BE}{}
{
  
\section{\textcolor{Chapter }{Utility Functions}}\label{Chapter_Utility_Functions_Section_Utility_Functions}
\logpage{[ 5, 1, 0 ]}
\hyperdef{L}{X810FFB1C8035C8BE}{}
{
  

 The following functions were created while implementing the high-level
functions, but we found that they could be useful outside of our functions,
too. You can find the specifications of the functions and some small examples
here. 

 

\subsection{\textcolor{Chapter }{IsMatrixInRREF}}
\logpage{[ 5, 1, 1 ]}\nobreak
\hyperdef{L}{X781CB82E87F343D1}{}
{\noindent\textcolor{FuncColor}{$\triangleright$\enspace\texttt{IsMatrixInRREF({\mdseries\slshape mat})\index{IsMatrixInRREF@\texttt{IsMatrixInRREF}}
\label{IsMatrixInRREF}
}\hfill{\scriptsize (function)}}\\
\textbf{\indent Returns:\ }
bool 



 Checks whether the matrix \emph{mat} is in RREF }

 
\begin{Verbatim}[commandchars=!@|,fontsize=\small,frame=single,label=Example]
  !gapprompt@gap>| !gapinput@M := RandomMat( 3, 3 );;|
  !gapprompt@gap>| !gapinput@Display( M );|
  [ [   1,   0,  -1 ],
    [  -1,  -1,   1 ],
    [  -1,   1,  -2 ] ]
  !gapprompt@gap>| !gapinput@IsMatrixInRREF( M );|
  false
  !gapprompt@gap>| !gapinput@L := [ [ 1, 0, 3 ], [ 0, 1, 5 ] ];;|
  !gapprompt@gap>| !gapinput@Display( L );|
  [ [  1,  0,  3 ],
    [  0,  1,  5 ] ]
  !gapprompt@gap>| !gapinput@IsMatrixInRREF( L );|
  true
\end{Verbatim}
 

 

\subsection{\textcolor{Chapter }{RandomEchelonMat}}
\logpage{[ 5, 1, 2 ]}\nobreak
\hyperdef{L}{X7D1323CD795E8632}{}
{\noindent\textcolor{FuncColor}{$\triangleright$\enspace\texttt{RandomEchelonMat({\mdseries\slshape height, width, rank, randomSeed, field})\index{RandomEchelonMat@\texttt{RandomEchelonMat}}
\label{RandomEchelonMat}
}\hfill{\scriptsize (function)}}\\
\textbf{\indent Returns:\ }
mat 



 Constructs a random matrix in RREF of the given height (number of rows), width
(number of columns) and rank over the given field. Using the same random seed
will recreate the same matrix. }

 
\begin{Verbatim}[commandchars=!@|,fontsize=\small,frame=single,label=Example]
  !gapprompt@gap>| !gapinput@# We use Mersenne twister as a random seed here.|
  !gapprompt@gap>| !gapinput@randomSeed := RandomSource(IsMersenneTwister, 42);;|
  !gapprompt@gap>| !gapinput@M := RandomEchelonMat(10, 10, 5, randomSeed, GF(5));;|
  !gapprompt@gap>| !gapinput@Display(M);|
  1 . . . . . 1 4 4 .
  . 1 . . . 1 3 2 4 3
  . . 1 . . . 3 . 1 .
  . . . 1 . 2 2 3 4 2
  . . . . 1 2 2 4 1 4
  . . . . . . . . . .
  . . . . . . . . . .
  . . . . . . . . . .
  . . . . . . . . . .
  . . . . . . . . . .
\end{Verbatim}
 }

 }

 \def\indexname{Index\logpage{[ "Ind", 0, 0 ]}
\hyperdef{L}{X83A0356F839C696F}{}
}

\cleardoublepage
\phantomsection
\addcontentsline{toc}{chapter}{Index}


\printindex

\immediate\write\pagenrlog{["Ind", 0, 0], \arabic{page},}
\newpage
\immediate\write\pagenrlog{["End"], \arabic{page}];}
\immediate\closeout\pagenrlog
\end{document}
